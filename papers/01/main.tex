\documentclass[12pt,letterpaper,oneside,twocolumn]{article}

\usepackage{times}
\usepackage{stix}
\usepackage{graphicx}
\usepackage{indentfirst} % Auto indent every paragraph
\usepackage{chemformula}  % provides \ch{H2O}
\usepackage{titlesec}  % provides \titlespacing
\usepackage{setspace} % provides \setstretch
% Below are ACS defaults
\usepackage[super,comma,numbers,sort&compress]{natbib}
\usepackage[margin=5pt,font=small,labelfont=bf]{caption}
\usepackage[margin=2.54cm]{geometry}
\usepackage{amsmath}
\usepackage{booktabs}

\frenchspacing    % don't use double spaces after periods, colons, etc.
\titlespacing\section{0pt}{12pt plus 4pt minus 2pt}{0pt plus 2pt minus 2pt}
\titlespacing\subsection{0pt}{12pt plus 4pt minus 2pt}{0pt plus 2pt minus 2pt}
\titlespacing\subsubsection{0pt}{12pt plus 4pt minus 2pt}{0pt plus 2pt minus 2pt}
\setstretch{1.05}

\usepackage{lipsum}
\begin{document}
\title{\vspace{-4ex}The Exa Framework\vspace{-1ex}}
\author{
	Thomas J. Duignan and Alex Marchenko\footnote{tjduigna@buffalo.edu; alexmarc@buffalo.edu} \\
	University at Buffalo, \\
	State University of New York, \\
	Buffalo, NY 142260-3000, USA \\
}
\date{\vspace{-1ex}\today}
\maketitle

\begin{abstract}
\lipsum[10]
\end{abstract}

\section*{Summary}
Data can have all sorts of structure; often it is in the form of n-dimensional 
arrays, i.e. dataframes and series in the Python pandas or R sense.
Typically multiple dataframes/series objects are the output of an experiment
and this data has implicit relationships. The container concept stores these 
data objects and keeps track of their relationships. The container provides
a way of organizing, processing, and visualization the different data objects
individually or the representation of the data collection as a whole (if 
applicable).
\lipsum[1]
\lipsum[5]
A typical computational workflow follows the following motif; 1) creation of input files and script, 2) transfer of these scripts to a computing resource (such as a supercomputer with a batch queue system), 3) confirmation of successful completion [lets assume that this passes], 4) transfer of output files, 5) post-processing and visualization of results.

Typical scientific workflow is to come up with idea, build hypothesis, perform experiments, analyze results, write paper, publish
\section*{Introduction}
\lipsum[2]
\section*{Editors}
\lipsum[3]
\section*{Containers}
\lipsum[4]
\section*{Workflows}
\lipsum[5]

\bibliographystyle{unsrtnat}
\bibliography{main}

\end{document}
